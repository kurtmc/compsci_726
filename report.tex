\documentclass[9pt,technote]{IEEEtran} 
\usepackage[numbers]{natbib}

\title{[Name of framework]: A Proposed Framework For Standardised Implementation of Countermeasures Against Attacks on Tor} 
\date{today}
\author{Jenna FinlaysonKurt McAlpine}

\begin{document} 
\maketitle

\begin{abstract} Tor provides anonymity by routing communications within layers
of encryption. A vulnerability in the Tor network is the indirect rate reduction
attack. This attack can reduce anonymity by monitoring client connections and
spoofing traffic on Tor exit relay. By spoofing messages to an exit relay at a
rate slower than the client was previously receiving messages, a rate reduction
can be observed. This indicates that the client is likely to be communicating
with that specific server. We propose a method to counteract this attack by
extending Tor to mask the rate of messages being received. \end{abstract}

\section{Introduction} 
The majority of surveillance that occurs today is the
surveillance of data and communications on the internet
\cite{diffie2008brave}. Since the 2013 leaks from Edward Snowden we now know the scale of and government power behind mass internet
surveillance. It is therefore important for internet users to have a way to
protect themselves from certain types of surveillance that can expose their identities and online behaviour. One solution to this is to use an anonymity network such as Tor. Tor allows anonymous communication by using layers of encryption and routing network
traffic between multiple volunteer operated onion routers \cite{tor}. This aids in the concealment of the
user's physical location, identity, and usage from potential surveyors. To keep
safe from government surveillance it is important to employ countermeasures for the security
vulnerabilities in Tor as they are identified.\\

New security attacks on Tor are being discovered frequently and are often published with specific ways to counteract them. Most of these proposed countermeasures come at a cost to the performance [and other things] of Tor and require implementations in the Tor source code directly. This requires knowledge of the Tor architecture and code base in order to implement the countermeasure, and then needs to be integrated and deployed in Tor itself. This process takes time and is not a simple process due to Tor's complexity so it therefore leaves Tor open to the identified vulnerabilities for potentially very long periods of time. \\

We propose [Name of Framework], a framework for standardised implementation of countermeasures against attacks on Tor. [overview of stuff about how our framework works].

% Here's all the citations we have to use. 
\cite{hayesguard}\cite{gilad2012spying}\cite{sun2015raptor}\cite{biryukov2012torscan}\cite{jansen2014sniper}\cite{tor}

\section{Related Work} In order to show the types of attacks that can be used to
reduce the anonymity of the Tor network, we summarise papers that identify some
of these types of attacks and the authors' proposed countermeasures for them. We
then identify flaws in the implementation and potential deployment problems of
these countermeasures. We could not find any literature about any existing
standardised frameworks like the one we propose and have therefore focused on
identifying different attack types that our framework can work with.\\

\citeauthor{gilad2012spying} have detailed a type of attack on Tor that uses
eavesdropping only on the client, while carrying out an active attack on the
exit relay of a Tor circuit in order to identify communicating peers.


\bibliographystyle{plain} 
\bibliography{references}

\end{document}
