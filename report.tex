\documentclass[9pt,technote]{IEEEtran}
\usepackage[numbers]{natbib}

\title{[Name of framework]: A Proposed Framework For Simple Implementation of Countermeasures Against Attacks on Tor}
\date{\today}
\author{Jenna Finlayson\\Kurt McAlpine}

\begin{document}
\maketitle

\begin{abstract}
Tor provides anonymity by routing communications within layers of encryption. A
vulnerability in the Tor network is the indirect rate reduction attack. This
attack can reduce anonymity by monitoring client connections and spoofing
traffic on Tor exit relay. By spoofing messages to an exit relay at a rate
slower than the client was previously receiving messages, a rate reduction can
be observed. This indicates that the client is likely to be communicating with
that specific server. We propose a method to counteract this attack by
extending Tor to mask the rate of messages being received.
\end{abstract}

\section{Introduction}

The majority of surveillance that occurs today is the surveillance of data and
communications on the internet\cite{diffie2008brave}. Since the 2013 leaks from
Edward Snowden we now know for certain the scale and government power behind
mass internet surveillance. So it is important for us to have a way to protect
ourselves from certain types of surveillance. One of the solutions is to use an
anonymity network such as the Tor. Tor allows anonymous communications by
encrypting and routing network traffic between multiple volunteer operated
onion routers. This conceals the users physical locations and usage from
potential surveyors. Too keep ourselves safe from government surveillance it is
important to reduce the vulnerabilities in Tor is they become discovered. One
such vulnerability found in Tor is something called a ``Rate Reduction Attack''
whereby TCP traffic can be spoofed on a Tor exit node and the location of a Tor
user can be identified\cite{gilad2012spying}.

% Here's all the citations we have to use.
\cite{hayesguard}\cite{gilad2012spying}\cite{sun2015raptor}\cite{biryukov2012torscan}\cite{jansen2014sniper}

\section{Related Work}
In order to show the types of attacks that can be used to reduce the anonymity of the Tor network, we summarise papers that identify some of these types of attacks and the authors' proposed countermeasures for them. We then identify flaws in the implementation and potential deployment problems of these countermeasures.\\

\citeauthor{gilad2012spying} have detailed a type of attack on Tor that uses


\bibliographystyle{plain}
\bibliography{references}

\end{document}
